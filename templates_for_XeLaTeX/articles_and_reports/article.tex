%*********************************************************************
%                Template for articles
%                By Alexandre Krispin and all the great TeX community
%                blog : http://niktarace.wordpress.com/
%*********************************************************************
%!TEX TS-program = xelatex
%!TEX encoding = UTF-8 Unicode
\documentclass[a4paper,11pt,]{scrartcl}
\usepackage{xltxtra} %include fontspec, xunicode, etc. For more
                     %details, texdoc xltxtra
%\usepackage{mathspec}
%\usepackage{polyglossia} %xetex version of the babel
                                %package. Since it is incomplete,
                                %french users have to use babel. For
                                %other users, replace french by your language
%\setmainlanguage{french}
%\setmainlanguage[variant=british]{english} %for english users. For
%more documentation, texdoc polyglossia.
%\usepackage{xspace}
\usepackage[frenchb]{babel}


%**********************************************************
%Comment or uncomment to select one of the following fonts
%**********************************************************
%\setmainfont{Charis SIL}
\setmainfont{URW Palladio L} %I prefer this font
%\setmainfont{Linux Libertine O}
%\setmainfont{Gentium Book Basic}
% \usepackage{fourier}
% \usepackage[scaled=0.85]{berasans}
% \usepackage[scaled=0.85]{beramono}

%***********************************************************
%For japanese input
\newfontfamily\jap[Scale=0.8]{IPA明朝} %If you wish to write in
                                %japanese, then use the command
                                %\jap. Here is an example : "Hello,
                                %here is a test : {\ja こんにちはテス
                                %トです}." Try to compile it.
%***********************************************************
%If you want to write in arabic from the right to the left load the
%following package :
%\usepackage{bidi}
%then, use the command \setRL to right from the right to the left. For more details, texdoc bidi.
%***********************************************************
%Other useful packages for a basic article
\usepackage{url} %for url
\usepackage{longtable} %for long tables
\usepackage{textcomp} %additional characters
\usepackage{amsmath} %for maths
\usepackage{xcolor} %for colors
\usepackage{graphicx}
% ********************************************************************
% Hyperreferences setup
%*******************************************************
\usepackage{hyperref}
\hypersetup{%
    colorlinks=true, linktocpage=true, pdfstartpage=3, pdfstartview=FitV,%
    % uncomment the following line if you want to have black links (e.g., for printing)
    %colorlinks=false, linktocpage=false, pdfborder={0 0 0}, pdfstartpage=3, pdfstartview=FitV,% 
    breaklinks=true, pdfpagemode=UseNone, pageanchor=true, pdfpagemode=UseOutlines,%
    plainpages=false, bookmarksnumbered, bookmarksopen=true, bookmarksopenlevel=1,%
    hypertexnames=true, pdfhighlight=/O,%hyperfootnotes=true,%nesting=true,%frenchlinks,%
    urlcolor=brown, linkcolor=blue, citecolor=green, %pagecolor=RoyalBlue,%
    %urlcolor=Black, linkcolor=Black, citecolor=Black, %pagecolor=Black,%
    pdftitle={Article},%the title
    pdfauthor={Alexandre Krispin},%your name
    pdfsubject={},%
    pdfkeywords={},%
    pdfcreator={pdfLaTeX},%
    pdfproducer={LaTeX with hyperref and classicthesis}%
}

%**********************************************************
%Appearance
%pagestyle{scrheadings}
%\usepackage{fancyheadings}
\usepackage[left=4cm,right=4cm,top=3cm,bottom=3cm,includeheadfoot]{geometry}
%\usepackage{scrpage2}\pagestyle{scrheadings}
\pagestyle{headings}

%*********************************************************
%Table of content
\renewcommand\contentsname{table of contents}
\addtokomafont{sectionentry}{\rmfamily\mdseries\scshape\lowercase}
\addtokomafont{section}{\rmfamily\mdseries\scshape\color{spot}\lowercase}
\addtokomafont{subsection}{\rmfamily\mdseries\itshape\color{spot}}
\makeatletter \def\@seccntformat#1{\protect\makebox[0pt][r]{\csname
    the#1\endcsname\hspace{\marglistsep}}} \makeatother
\hypersetup{linkcolor=black,urlcolor=black}

%********************************************************
%Opening
\title{Analyse des origines de la Grande peur} %Enter the title of your article
\author{Alexandre Krispin} %Enter your name and first name

\begin{document}
%\frenchspacing
\maketitle
\tableofcontents

\section{Introduction}

L'année de 1789 marque la fin de l'Ancien Régime. Il s'agit autant du début de la Révolution que d'une crise qui marque la monarchie absolue par son caractère violent. Elle a marqué le clergé, la noblesse et le tiers-état. Ces trois ordres, n'ont rien en commun, mais en 1789, il semblerait qu'ils aient eu au moins en commun une chose : la peur.

D'après l'encyclopédie Universalis, la \og grande peur\fg~est une qualification recoupant différentes peurs, dont celle associée au complot aristocratique à la veille du 14 juillet 1789, celle associée aux partageux et anarchistes au moment de Brumaire 1799, mais il semblerait que le terme de Grande Peur a été réservé aux insurrections paysannes de 1789 \footnote{\emph{Cf.}, \url{http://www.universalis.fr/encyclopedie/grande-peur/}, consulté le 29/05/2010.} Nous verrons dans notre dossier ce qu'il en est plus exactement en essayant de comprendre comment la grande peur est née, à travers une analyse de différents facteurs.

Par grande peur, nous entendons la peur née durant l'année 1789. D'après \textsc{Lefebvre}, elle a été mal comprise par ses contemporains qui auraient été déconcertés, et qui aurait apparut comme un mystère, mal interprétée et mal analysée\footnote{\textsc{Lefebvre} Georges, \emph{La Grande peur de 1789}, Paris, éd. Librairie Armand Colin, 1970, p. 1.}. Au cours de notre analyse, nous nous servirons des connaissances acquises au travers de quelques livres, en nous penchant tout d'abord sur les facteurs objectifs, puis subjectifs de la grande peur.
\section{Les causes de la grande peur}
%\secttoc

\subsection{Les facteurs objectifs}

Il nous semble juste de considérer qu'il existe plusieurs facteurs qui peuvent expliquer la grande peur. Nous avons choisi de les diviser en deux catégories : les facteurs objectifs, et les facteurs subjectifs. Nous nous concentrerons dans cette sous-section sur les facteurs objectifs, et en particulier, dans un premier temps, sur un aspect important pour mieux comprendre ce qui nous intéresse, la loi de l'offre et de la demande concernant le froment. Puis dans un deuxième temps, nous aborderons une autre cause objective à la grande peur, la pauvreté.

\subsubsection{Le jeu de l'offre et de la demande}

Si l'on considère que le prix du blé est le \og baromètre de la vie
sociale\fg\footnote{\textsc{Guignet} Philippe, \textsc{Grevet} René,
  \emph{La France et les Français au XVIII\ieme~siècle (1715-1788), économie
et culture}, coll. \og Documents $\Sigma$ Histoire\fg, Paris, éd. Ophrys, 1993, p. 43.}, alors il nous semble pertinent d'observer dans un premier temps l'évolution des prix moyens nationaux annuels de l'hectolitre de froment par année civile de 1726 à 1790 :
\begin{longtable}{|l|c|r|}
\hline
année & prix moyens de l'hectolitre (en livres tournois)\\
\hline
1726 & 11.33\\
1727 & 9.23\\
1728 & 8.19\\
1729 & 9.05\\
1730 & 9.07\\
1731 & 9.38\\
1732 & 8.28\\
1733 & 8.07\\
1734 & 8.25\\
1735 & 8.11\\
1736 & 9.03\\
1737 & 9.42\\
1738 & 10.33\\
1739 & 11.36\\
1740 & 12.25\\
1741 & 14.18\\
1742 & 10.69\\
1743 & 7.82\\
1744 & 7.57\\
1745 & 7.62\\
1746 & 9.39\\
1747 & 12.04\\
1748 & 13.72\\
1749 & 12.46\\
1750 & 11.49\\
1751 & 11.67\\
1752 & 13.25\\
1753 & 11.85\\
1754 & 11.17\\
1755 & 8.54\\
1756 & 9.58\\
1757 & 11.89\\
1758 & 11.27\\
1759 & 11.76\\
1760 & 11.77\\
1761 & 10\\
1762 & 9.91\\
1763 & 9.53\\
1764 & 10.01\\
1765 & 11.16\\
1766 & 13.27\\
1767 & 14.32\\
1768 & 15.51\\
1769 & 15.38\\
1770 & 18.82\\
1771 & 18.16\\
1772 & 16.65\\
1773 & 16.44\\
1774 & 14.57\\
1775 & 15.89\\
1776 & 12.91\\
1777 & 13.36\\
1778 & 14.67\\
1779 & 13.59\\
1780 & 12.59\\
1781 & 13.45\\
1782 & 15.26\\
1783 & 15.02\\
1784 & 15.33\\
1785 & 14.83\\
1786 & 14.13\\
1787 & 14.16\\
1788 & 16.09\\
1789 & 21.92\\
1790 & 19.45\\
 \hline
 \caption{Tableau des prix moyens nationaux annuels de l'hectrolitre de froment par année civile de 1726 à 1790 (source : \textsc{Guignet} Philippe, \textsc{Grevet} René,
  \emph{La France et les Français au XVIII\ieme~siècle (1715-1788), économie
et culture}, coll. \og Documents $\Sigma$ Histoire\fg, Paris, éd. Ophrys, 1993, p. 46-47.)}
\end{longtable}

Clairement, nous pouvons distinguer dans l'évolution des prix moyens de l'hectolitre de froment une très forte hausse en 1789. À cela il semblerait juste de prendre en compte plusieurs facteurs :
\begin{description}
 \item[La démographie] Au XVIII\ieme~siècle, la France a connu un essor démographique. Entre 1749 et 1778, la population augmente d'environ 3 200 000 habitants\footnote{\textsc{Guignet} Philippe, \textsc{Grevet} René, \emph{La France et les Français au XVIII\ieme~siècle (1715-1788), économie et culture}, coll. \og Documents $\Sigma$ Histoire\fg, Paris, éd. Ophrys, 1993, p. 53.}. Cette augmentation démographique serait une cause à l'augmentation globale du prix de l'hectolitre tout au long du siècle. Cette hypothèse s'appuie sur l'idée de la régulation du marché par la loi de l'offre et de la demande ; si, avec l'augmentation de la demande, l'offre n'augmente pas non plus, alors les prix augmentent. Toutefois, nous ne nous avancerons pas en parlant d'inflation comme le fait Philippe \textsc{Guignet}, sans avoir donné l'évolution des prix moyens des autres marchandises, puisqu'il faut rappeler qu'\og il ne suffit pas que le prix d'un seul bien augmente pour qu'il y ait inflation. Il faut que les prix de \emph{la plupart} des biens augmentent\fg\footnote{\textsc{Stiglitz} Joseph E., \textsc{Walsh} Carl E., \textsc{Lafay} Jean-Dominique, \emph{Principes d'économie moderne}, [3\ieme~édition], Bruxelles, éd. De Boeck Université, 2007, p. 484.}
 \item[La physiocratie] Vers les années 1750, un courant de pensée en économie s'est développé, la physiocratie. Les physiocrates stipulent qu'il faut une libéralisation du commerce des grains. Pour \textsc{Quesnay} par exemple, le seul moyen d'augmenter la production était de laisser les prix augmenter.\footnote{\textsc{Guignet} Philippe, \textsc{Grevet} René,
  \emph{La France et les Français au XVIII\ieme~siècle (1715-1788), économie
et culture}, coll. \og Documents $\Sigma$ Histoire\fg, Paris, éd. Ophrys, 1993, p. 49.} Ainsi, le courant de pensée d'alors tend à considérer cette hausse généralede l'helctolitre comme étant normal.
 \item[Le \og jeu du hasard météorologique\fg] À plusieurs reprises, des années climatiquement catastrophiques se sont multipliées après 1765. Il est d'ailleurs attesté qu'il y a eu un refroidissement des étés et des printemps durant les dix dernières années de l'Ancien Régime. En 1784 et 1785, la sécheresse a égalemet frappé plusieurs régions. Or, des mauvaises récoltes dûes à des désastres climatiques impliquent nécessairement la montée des prix sur le marché de l'offre et la demande. La montée des prix, lorsqu'elle n'est pas contrôlée comme cela a été le cas dans la deuxième partie du XVIII\ieme~siècle, entraîne des émeutes. Comme l'écrivait d'ailleurs \textsc{Lefebvre}, \og en temps de disette, la faim provoquait aussi l'émeute, laquelle, à son tour, suscitait ou fortifiait la peur.\fg\footnote{\textsc{Lefebvre} Georges, \emph{La Grande peur de 1789}, Paris, éd. Librairie Armand Colin, 1970, p. 27.}
\end{description}
Aussi pertinents ces facteurs soient-ils, ils ne sont cependant pas les seuls. En l'occurrence, un autre facteur objectif qui nous paraît être fondamental, qui relève également de l'économie, mais qui possède aussi un enjeu social et politique, est la pauvreté.

\subsubsection{Pauvreté et bien-être au XVIII\ieme~siècle}

\paragraph{PIB et bien-être de la société}
\textsc{Lermarchand} écrivait que \og récemment Chr. \textsc{Morisson} (2007) a présenté une estimation nouvelle de l'évolution du PIB de la France pendant XVIII\ieme~siècle qui va dans le sens de l'optimisme.\fg\footnote{\textsc{Lemarchand} Guy, \emph{L'économie en France de 1770 à 1830} \textemdash~De la crise de l'Ancien Régime à la révolution industrielle, coll. U, Paris, éd. Armand Colin, 2008, p. 103.} D'autre part, le chiffre de croissance serait positif entre 1788 et 1790 pour l'agriculture\footnote{\emph{Ibid.}, p. 104.}, bien que les chiffres présentés restent conjecturaux. Toujours est-il que, bien que le PIB puisse être considéré comme\og la meilleure estimation disponible du niveau de la
production résultant de l'activité de marchés\fg\footnote{\textsc{Stiglitz} Joseph E., \textsc{Walsh} Carl E., \textsc{Lafay} Jean-Dominique, \emph{Principes d'économie moderne}, [3\ieme~édition], Bruxelles, éd. De Boeck Université, 2007, p. 456.}, il n'en reste pas moins qu'il \og n'est qu'une mesure partielle du bien-être total la société\fg\footnote{\emph{Ibid.}, p. 456.} En l'occurrence, le PIB, qu'il soit positif ou non, ne peut nous permettre de déduire les conditions de l'époque. Au contraire, alors que le PIB semble avoir été positif dans les années 1788 et 1790, la pauvreté semble avoir atteint son plus au point.

\paragraph{Crise du système clérical}
Avec le gouvernement révolutionnaire sera créé un Comité de la mendicité, présidé par La \textsc{Rochefoucauld-Liancourt}, signe d'une prise de conscience que la pauvreté est un problème dans la société, et non un problème en dehors, qui ne concernerait qu'une minorité. Mais avant cette prise de conscience, les indigents étaient voués à eux-mêmes, avec pour seule aide l'Église et la charité. En effet, \og c'est l'Église qui assume la responsabilité de l'organisation quotidienne des secours sous l'Ancien Régime. La plupart des hôpitaux sont des fondations cléricales, créées grâce à des dons qui remontent à une époque lointaine et toujours gérées selon des principes strictement chrétiens.\fg\footnote{\textsc{Forrest} Alan, Revellat Marie-Alix (trad.), \emph{La Révolution française et les pauvres}, éd. Librairie Académique Perrin, 1986, p. 39.} Cependant la fin de l'Ancien Régime marque également une difficulté croissante pour jouer son rôle de charité ; à cela, deux raisons :
\begin{enumerate}
 \item une diminution de la foi de plus en plus généralisée faisait diminuer inévitablement les dons et donc les fonds de l'Église pour l'exercice de la charité\footnote{Voir \textsc{Forrest} Alan, \textsc{Revellat} Marie-Alix (trad.), \emph{La Révolution française et les pauvres}, éd. Librairie Académique Perrin, 1986, p. 16 et p. 41.}
 \item l'ampleur du problème de la pauvreté, qui était devenu un véritable problème social et de fait une forte disparité entre les fonds de l'Église et les besoins.
\end{enumerate}

\paragraph{La pauvreté, phénomène répandu} D'après le 5\ieme~rapport du Comité de Mendicité, la proportion des pauvres est de 1\textfractionsolidus 7 à Soissons, 1\textfractionsolidus 6 à Montauban, et 1\textfractionsolidus 10 à Metz, ce qui est naturellement considérable. Ces statistiques font surgir une question triviale : qu'est-ce que la pauvreté ? Nous n'avons pas eu le temps de traiter cette question dans notre devoir, bien que selon \textsc{Forret} il semblerait qu'il y ait une certaine subjectivité dans la définition de la pauvreté. Toutefois, la pauvreté, quoique relative ainsi que le montrait Denis Clerc est des psychologues, ne pas atteindre la norme de la société, \og c'est se marginaliser, c'est se paupériser. Car, à la différence de la misère (qui est une insuffisance absolue), la pauvreté est relative.\fg\footnote{\textsc{Clerc} Denis, \emph{D\'echiffrer l'\'economie}, Paris, La D\'ecouverte, 2007, p. 61.} Ainsi, quoique la pauvreté ait pû être relative, et la misère mal définie à la fin du XVIII\ieme~siècle, il n'en reste pas moins que la réalité sociale est ce que les individus pensent d'elle, comme ils la désignent et peu importe la conformité au canon d'une définition\footnote{\emph{Cf.}, \textsc{Fischer} Gustave-Nicolas, \emph{Psychologie de l'environnement social}, Paris, éd. Dunod, 1996, p. 162.} ; ce qui nous amène à prendre en compte les facteurs subjectifs de la grande peur.

\subsection{Les facteurs subjectifs}

\subsubsection{La faim}

\textsc{Taine} écrivait que le peuple \og ressemble à un homme qui marcherait dans un étang, ayant de l'au jusqu'à la bouche ; à la moindre dépression du sol, au moindre flot, il perd pied, enfonce et suffoque.\fg\footnote{\textsc{Lefebvre} Georges, \emph{La Grande peur de 1789}, Paris, éd. Librairie Armand Colin, 1970, p. 7.} En effet, il semblerait juste de penser d'après les études historiographiques de \textsc{Lefebvre} que la majorité des Français vivait dans la crainte de la faim, que c'était une de leur plus grande ennemi. Or une ennemi dont on ne sait quand elle peut attaquer et potentiellement une angoissante de tout moment. Aussi, la menace de la faim a sans doute participé à une angoisse générale.

Dans le contexte d'une pauvreté généralisée et d'une mauvaise récolte il semblerait alors que la manifestation d'une panique générale, soit non seulement justifiée, mais également logique et inévitable. D'autre part, la faim engendrait naturellement la mendicité.\footnote{\textsc{Lefebvre} Georges, \emph{La Grande peur de 1789}, Paris, éd. Librairie Armand Colin, 1970, p. 15.} [...] Or comme nous l'avons vu, l'Église faisait face à une crise systémique due au nombre de croyant réduit, à une diminution de l'aumône, des dons. Ceci a donc eu pour conséquence le fait qu'\og il n'y avait pas de secours pour le chômeur\fg\footnote{\emph{Ibid.}, p. 15.}, ou en tous les cas un secours de moins en moins présent. Or cela devait avoir un impact sur la société en terme de violences sociales.

\subsubsection{La peur des indigents}

Tandis que la faim gagnait les indigents avec la mauvaise récolte de 1789, les indigents se laissaient aller naturellement à leurs instincts. Or avec une mauvaise récolte, le nombre d'indigent ne peut qu'augmenter. C'est pourquoi durant cette année, avec les attroupements de mendiants, d'errants, dont le but était de se nourrir, ou de satisfaire ses envies primaires, ils ne pouvaient que faire peur au paysan moyen qui pouvait encore vivre et manger convenablement. Lefebvre notait par exemple qu'aux environs de Chartres, on relatait, le 24 juillet 1789 :
\begin{quotation}
 L'esprit de la populace paraît actuellement si échauffé que, consultant le besoin présent et pressant, il peut se croire autorisé à soulager sa misère, lorsque la récolte s'ouvrira. Non seulement les glanes, son patrimoine ordinaire, seront l'objet de son empressement, mais, réduit aux abois par une cherté excessive et longue, il pourra se dire : Dédommageons-nous de la misère passée ; tout en commun dans l'extrême nécessité ; mangeons à notre faim... Cette expédition populaire équivaudrait bien le fléau de la grêle.\footnote{\emph{Ibid.}, p. 20.}
\end{quotation}

Toutefois il faut nuancer notre propos. Certes, le paysan était effrayé à partir du moment où ses intérêts (sa récolte) étaient en jeu. Mais il était lui-même plus proche d'une situation d'indigence que de la situation d'un seigneur ou d'un grand bourgeois. C'est pourquoi les mendiants n'étaient pas nécessairement entourés d'une incompréhension de la part des paysans, mais seulement et essentiellement d'une peur pour leurs intérêts. \textsc{Lefebvre} notait par exemple que certains cahiers protestent même contre l'internement des mendiants dans des maisons de force.\footnote{\emph{Ibid.}, p. 15.}

Toujours est-il que dans la société précédent la Révolution, la peur était endémique. Cette peur  est également liée fortement avec les divers théories du complot qui ont surgit en particulier en 1789.

\subsubsection{Les théories du complot}

En 1789, le tiers-état était convaincu qu'il y avait un complot contre lui de la part de l'aristocratie visant à l'exterminer. La situation économique de la France à cette date là a créé un terrain favorable à l'émergence de diverses théories du complot. D'une part, il faut souligner que la situation économique qui s'accompagnait d'une recrudescence de mendiants, de pauvres cherchant à satisfaire leurs besoins par n'importe quel moyent, un sentiment d'insécurité avait émergé. Cependant étant donné l'ampleur de la grande peur, il serait sans doute pertinent de penser que l'insécurité, l'angoisse, étaient enracinés profondément, et que la mauvaise récolte de 1789 n'a fait qu'immerger si ce n'était pas déjà le cas des émotions déjà présente avant.

D'autre part, la panique populaire et la théorie largement répandue d'un complot tendant à affamer le peuple était apparu dans un climat de méfiance de la part des paysans vis-à-vis des aristocrates. Or, d'où venait cette méfiance ? Il nous semble pertinent de penser qu'elle venait essentiellement du despotisme éclairé prôné par la monarchie absolue, qui prenait des décisions sur la base d'une idéologie physiocratique sans se soucier de l'avis du tiers-état. Lefebvre écrit par exemple que :
\begin{quotation}
 Jamais le peuple n'admettait que la nature fût seule responsable de sa misère. Pourquoi, dans les années fécondes, n'avait-on pas mis de blé en réserve ? C'est que les riches, propriétaires et fermiers, de connivence avec les marchands et avec la complicité des ministres et autres hommes du roi, toujours favorables aux puissants, avaient exporté les excédents pour les vendre au loin à bon prix.\footnote{\emph{Ibid.}, p. 27}
\end{quotation}
Ainsi les théories du complot nous semblent avoir pour origine une mauvaise gestion de l'économie et de la pauvreté, ce qui a sans doute poussé le tiers-état à être opposé au gouvernement de Louis XVI. C'est, nous semble-t-il, cette opposition qui s'est mutée en une peur irrationnelle invoquant le complot de l'aristocratie comme justification de cette peur.

\section{Conclusion}
  
  Dans un premier temps, nous avons retracé différents facteurs objectifs pouvant expliquer les origines de la grande peur ; ce sont en particulier des facteurs liés à la loi de l'offre et de la demande selon laquelle les prix du froment avait augmenté, et liés à la pauvreté. Puis dans un deuxième temps, nous avons essayé de mettre en avant quelques facteurs subjectifs qui nous paraissaient essentiels. À savoir la peur de la faim, des indigents, mais également une peur mutuelle ou \og antagoniste\fg~entre noblesse et tiers-état, entre aristocrates et paysans.
  
  Notre dossier montre clairement, nous semble-t-il, le fait que l'on ne peut pas associer une peur populaire avec une complète irrationnalité ; il y a bien des facteurs objectifs à cela. Mais la présence d'une certaine subjectivité montre également que malgré toutes les mesures que l'on pourrait faire pour montrer la richesse de la France durant cette période révolutionnaire, le bien-être n'était pas à son comble, et que la révolution était de fait inévitable.
  
  Ainsi, le bien-être dans une société n'est pas tant lié à des critères tels que le PIB qu'à la pauvreté. En ce sens, bien que le Comité de Mendicité n'ait que peu servit de manière pratique, les résultats de leur travail sont intéressants : la pauvreté est l'affaire de la société, et l'éliminer et la condition au bien-être social qui, lui-même, est un des principaux facteurs permettant à la richesse d'une nation de prospérer.
  
 \newpage
\begin{thebibliography}{7}
\bibitem[1]{Clerc07}
\textsc{Clerc} Denis, \emph{D\'echiffrer l'\'economie}, Paris, La D\'ecouverte, 2007,
\bibitem[2]{Fischer96}
\textsc{Fischer} Gustave-Nicolas, \emph{Psychologie de l'environnement social}, Paris, éd. Dunod, 1996.
\bibitem[3]{Forrest86}
\textsc{Forrest} Alan, \textsc{Revellat} Marie-Alix (trad.), \emph{La Révolution française et les pauvres}, éd. Librairie Académique Perrin, 1986.
\bibitem[4]{Guignet93}
\textsc{Guignet} Philippe, \textsc{Grevet} René, \emph{La France et les Français au XVIII\ieme~siècle (1715-1788), économie et culture}, coll. \og Documents $\Sigma$ Histoire\fg, Paris, éd. Ophrys, 1993.
\bibitem[5]{Lefebvre70}
\textsc{Lefebvre} Georges, \emph{La Grande peur de 1789}, Paris, éd. Librairie Armand Colin, 1970.
\bibitem[6]{Lemarchand08}
\textsc{Lemarchand} Guy, \emph{L'économie en France de 1770 à 1830} \textemdash~De la crise de l'Ancien Régime à la révolution industrielle, coll. U, Paris, éd. Armand Colin, 2008.
\bibitem[7]{Stiglitz07}
\textsc{Stiglitz} Joseph E., \textsc{Walsh} Carl E., \textsc{Lafay} Jean-Dominique, \emph{Principes d'économie moderne}, [3\ieme~édition], Bruxelles, éd. De Boeck Université, 2007.
\end{thebibliography}

\end{document}
