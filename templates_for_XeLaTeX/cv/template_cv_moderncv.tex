%Time-stamp: <2010-09-28 17:18:48 (freeman)>
\documentclass[11pt,a4paper]{moderncv}
% moderncv themes
\moderncvtheme[green]{casual} % optional argument are 'blue' (default), 'orange', 'red', 'green', 'grey' and 'roman' (for roman fonts, instead of sans serif fonts)
% \moderncvtheme[blue]{classic} % idem
\usepackage{xltxtra}

% HEADINGS
%\usepackage{sectsty} 
\usepackage[normalem]{ulem} 
\sectionfont{\rmfamily\mdseries\upshape\Large}
\subsectionfont{\rmfamily\bfseries\upshape\normalsize} 
%\subsubsectionfont{\rmfamily\mdseries\upshape\normalsize}
% FONTS
%\setromanfont [Ligatures={Common}, BoldFont={Gentium Basic Bold}, ItalicFont={Gentium Basic Italic}]{Gentium Basic}
%\setmonofont[Scale=0.8]{Monaco}
%\setmainfont[Numbers=OldStyle,Ligatures=Historic]{TeX Gyre Pagella}
%For japanese input
\newfontfamily\ja[Scale=0.8]{IPA明朝}

\usepackage{xcolor} %character encoding

\usepackage[frenchb]{babel}
% adjust the page margins
\usepackage[scale=0.8]{geometry}
% \setlength{\hintscolumnwidth}{3cm} % if you want to change the width of the column with the dates
% \AtBeginDocument{\setlength{\maketitlenamewidth}{6cm}} % only for the classic theme, if you want to change the width of your name placeholder (to leave more space for your address details
% \AtBeginDocument{\recomputelengths} % required when changes are made to page layout lengths

% personal data
\firstname{Alexandre} \familyname{Krispin}
% title{Resum\'e title
%   (optional)} % optional, remove the line if not wanted
\address{41 rue Guynemer}{93200 St
  Denis} % optional, remove the line if not wanted
\mobile{0628766934} % optional, remove the line if not wanted
% \phone{phone (optional)} % optional, remove the line if not wanted
% \fax{fax (optional)} % optional, remove the line if not wanted
\email{k.m.alexandre@gmail.com} % optional, remove the line if not wanted
\homepage{http://alexkrispin.wordpress.com/} % optional, remove the line if not wanted
\extrainfo{\begin{center}
\scriptsize Dernière mise à jour : \today\ -- Compilé avec \fontspec{Times New
Roman}\XeLaTeX\end{center}} % optional, remove the line if not wanted
% \photo{photo.png} % '64pt' is the height the picture must be resized to and 'picture' is the name of the picture file; optional, remove the line if not wanted
% \quote{BTCQB (baise tout ce qui bouge XD XD ---> voir Elie
%   Semoun)} % optional, remove the line if not wanted

% to show numerical labels in the bibliography; only useful if you
% make citations in your resume
\makeatletter
\renewcommand*{\bibliographyitemlabel}{\@biblabel{\arabic{enumiv}}}
\makeatother

% bibliography with mutiple entries \usepackage{multibib}
% \newcites{book,misc}{{Books},{Others}}

% \nopagenumbers{} % uncomment to suppress automatic page numbering for CVs longer than one page
% ----------------------------------------------------------------------------------
% content
% ----------------------------------------------------------------------------------
\begin{document}
\maketitle

\section{Formation}
\cventry{2009--2010}{Licence}{Universit\'e Paris 8}{St
  Denis}{\textit{2\ieme\ ann\'ee}}{Science
  politique} % arguments 3 to 6 can be left empty
\cventry{2007--2009}{Master de
  Japonais}{INALCO}{Paris}{\textit{5\ieme\ ann\'ee}}{Sp\'ecialit\'e
  civilisation japonaise} \cventry{2004}{Baccalaur\'eat
  Scientifique}{Lyçée La
  Versoie}{Thonon-les-Bains}{\textit{Terminal}}{sp\'ecialit\'e
  Physique-chimie}

\section{M\'emoire de Master}
\cvline{Titre}{\emph{La notion de \emph{ren} {\jap 仁}
    \`a la fin de l'\'epoque d'Edo et ses enjeux socio-politiques}}
\cvline{Superviseur}{Fran\c cois Mac\'e} \cvline{description}{\small
  J'\'etudie dans une premi\`ere partie des concepts
  confuc\'eens. Dans une deuxi\`eme partie, j'\'etudie \`a partir de
  ces m\^emes concepts les causes d'une r\'ebellion organis\'ee lors
  d'une famine de 1835-37. Puis je montre les solutions envisag\'ees
  par les penseurs de l'\'epoque pour lutter contre les famines et les
  d\'esordres socio-\'economiques.}

\section{Exp\'eriences professionnelles}
\cventry{avril 2010}{Magasinier \`a
  l'IHEJ}{Coll\`ege-de-France}{Paris}{}{} \cventry{juillet--ao\^ut
  2005}{Magasinier}{Gedimat}{Thonon-les-Bains}{}{}

\section{Langues}
\cvlanguage{Anglais}{lu, \'ecrit, parl\'e}{} \cvlanguage{Japonais}{lu,
  \'ecrit, parl\'e}{}

\section{Informatique}
\cvcomputer{Bureautique}{Microsoft Office, \LaTeX, \XeLaTeX, OpenOffice.org,
  KOffice, Gnumeric}{Syst\`emes d'exploitation}{GNU/Linux, Mac OSX,
  FreeBSD, Microsoft Windows} \cvcomputer{Graphisme}{Gnuplot,
  Illustrator}{}{} \cvcomputer{\'Editeurs de texte}{Vim, Emacs}{}{}

\section{Centres d'int\'er\^ets}
\cvline{Sports}{\small j\=ud\=o, j\=ujits\=u, boxe}
\cvline{Loisirs}{\small musique, lecture}

% \section{Extra 1}
% \cvlistitem{Item 1} \cvlistitem{Item 2} \cvlistitem[+]{Item
%   3} % optional other symbol

% \renewcommand{\listitemsymbol}{-} % change the symbol for lists

% Publications from a BibTeX file without
% multibib\renewcommand*{\bibliographyitemlabel}{\@biblabel{\arabic{enumiv}}}% for BibTeX numerical labels
% \nocite{*} \bibliographystyle{plain}
% \bibliography{publications} % 'publications' is the name of a BibTeX file

% Publications from a BibTeX file using the multibib package
% \section{Publications}
% \nocitebook{book1,book2} \bibliographystylebook{plain}
% \bibliographybook{publications} % 'publications' is the name of a BibTeX file
% \nocitemisc{misc1,misc2,misc3} \bibliographystylemisc{plain}
% \bibliographymisc{publications} % 'publications' is the name of a BibTeX file

\end{document}

\endinput
